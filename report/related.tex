% !TEX root =  main.tex

\section{Previous Work}
    \todo{finish plz. Mainly this is Vuduc}Vuduc thesis~\cite{vuduc04}
Fill estimation is an important intermediate step for matrix computation optimization. Specifically, it is commonly used in autotuning, or the process of the optimal blocking scheme for sparse matrix-vector multiplication~\cite{WillOlVu09, VuducDeYe02, YilmAkGa16, VuducMo05}. Autotuning for matrix computations on GPUs~\cite{ChoiSiVu10}.

Code and algorithm we benchmark against provided in the {OSKI} sparse matrix library~\cite{VuducDeYe05}.

performance tuning for sparse matrix kernels~\cite{ImYeVu04, ImYe01}. Performance is generally weak on modern parallel computers, but can do performance engineering in the single-system case. 
sparse matrix


{Sparsity} - optimizing 

\subsection{Heuristic for fill estimation}

To our knowledge, there are no guarantees on the accuracy existing algorithms for fill estimation. The best known heuristic (due to Vuduc~\cite{vuduc04}) for computing the fill uses randomization, and does not calculate the optimal fill for offsets --- it only estimates the fill for block size pairs $(r,c)$. Furthermore, there is only an empirical study of the cost-accuracy tradeoff in fill estimation. At a high level, the algorithm tests all possible block sizes $(r,c)$ by looping through all possible row dimensions $r$ and dividing the matrix into $m/r$ ``row columns'', each composed of $r$ consecutive row. That is, the $i$-th ``row column'' is rows $ri$ through $r(i+1)$.

For a fixed $r$, the algorithm randomly chooses (with probability $\sigma$) to evaluate a row. That is, it flips a coin (potentially biased) to determine whether or not to process a row. In order to process a row, it processes all non-zero elements by setting a relevant bit to keep track of which blocks have already been counted. The algorithm also counts how many nonzeros it has processed and the number of non-empty blocks found in each possible blocking scheme.

The heuristic returns the maximum fill as the sample optimum blocking scheme.

 \begin{samepage}
    \begin{alg}
      Given a sparse matrix of dimensions $m \times n$ and $B$, estimate the fill by evaluating a subset of the rows in each matrix
      \begin{algorithmic}[1]
        \Function{Oski Heuristic}{matrix $A$ (dimensions $m \times n$), max block size $B$, evaluation probability $\sigma$} \\
       \State \texttt{max\_fill} $\gets$ max fill
       \State \texttt{best\_pair} $\gets \emptyset$
          \For{$r = 1 \to B$}
          \State array \texttt{num\_blocks[1:$B$]} $\gets$ 0
          \State \texttt{nnz\_visited} $\gets$ 0
          \For{i from [1, m/r]}
          \State flip a coin with heads probability $\sigma$
          \State array \texttt{last\_block\_index}[1:B] $\gets -1$
          \If the coin came up heads
          \For{nonzero $A(i,j) \in A(ir, (i+1)r-1)$}
          \State \texttt{nnz\_visited} $\gets$ \texttt{nnz\_visited} + 1
          \For{$c = 1 \to B$}
          \State \texttt{last\_block\_index}[c] $\gets j/c$
          \State array \texttt{num\_blocks}[c] $\gets$ \texttt{num\_blocks}[c] + 1
          \EndFor
          \EndFor
          \State $\hat{f}_{r,c}(A, \sigma) \gets rc\texttt{num\_blocks}[c]  /  \texttt{nnz\_visited} $
          \EndIf
          \EndFor
          \EndFor
        \EndFunction
      \end{algorithmic}
      \label{alg:oski}
    \end{alg}
    \end{samepage}


\subsection{Asymptotic Runtime Analysis}
