% !TEX root =  main.tex

\section{Notation}
    A \textit{tensor} is a multidimensional array. A tensor of \textit{order} $N$ is an element of the tensor (direct) product of $N$ vector spaces. We assume all of our vector spaces are over an arbitrary field $\F$. Vectors are order 1 tensors and will be denoted with boldface lowercase letters, like this: $\vec{a}$. Matrices are order 2 tensors and will be denoted by boldface capital letters, like this: $\Mat{A}$. Tensors will be denoted by boldface capital Euler script letters, like this: $\Ten{A}$.

    We refer to populations using capital Euler script letters, like this: $\Pop{X}$. We refer to random variables and index bounds using capital letters, like this: $X$. We refer to functions, indices and elements of populations using lowercase letters, like this $x$.

    The $n^{th}$ element in a sequence is denoted by $\Ten{A}^{(n)}$.
    Element $(i_1, i_2, ..., i_N)$ of an order-$N$ tensor $\Ten{A} \in \F^{I_1 \times I_2 \times ... \times I_N}$ is denoted $ \Ten{A}[i_1, i_2, ..., i_N]$.\todo{ or $\Ten{A}_{i_1, i_2, ..., i_N}$.}
    Sometimes it is convenient to represent an $N$-dimensional index $i_1, i_2, ..., i_N$ as a vector, like this: $\vec{i}$.

    If we wish to represent the integer range $i, i + 1, ..., i'$, we use the syntax $i \to i'$. If we wish to represent the range of indices between two vectors, we use the syntax $\vec{i} \to \vec{i}'$, meaning $i_1 \to i_1', ..., i_N \to i_N'$.

    Subarrays are formed when we fix a subset of indices. We use a colon to indicate all elements of a mode. Thus, the middle $n/2$ columns of a matrix $\Mat{A} \in \F^{n \times n}$ would be written $\Mat{A}_{:, n/4 \to 3n/4}$.

    \todo{For convenience, we say that $\Ten{A} = 0$ if and only if every element of $\Ten{A}$ is 0.}
